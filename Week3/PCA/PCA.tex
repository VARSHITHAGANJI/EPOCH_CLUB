\documentclass[12pt,letterpaper, onecolumn]{exam}
\usepackage{amsmath}
\usepackage{amssymb}
\usepackage{sidecap}
\usepackage{tabularx}
\usepackage{csquotes}
\usepackage{makecell}
\usepackage{hyperref}
\hypersetup{
    colorlinks=true,
    linkcolor=blue,
    filecolor=magenta,      
    urlcolor=black,
    pdftitle={Overleaf Example},
    pdfpagemode=FullScreen,
}
%\usepackage[left=0.5cm,right=0.5cm,top=0.5cm,bottom=0.5cm]{geometry}
\usepackage[usestackEOL]{stackengine}
%\setstacktabbedgap{1ex} 
\usepackage{tikz}
\usetikzlibrary{decorations.pathreplacing}
\usetikzlibrary{fadings}
\def\layersep{2.5cm}

\usepackage{enumitem}
\usepackage{graphicx}
\usepackage{algorithm}
\usepackage{algpseudocode}

\usepackage{subcaption}

%\usepackage[shortlabels]{enumitem}
%\usepackage{enumerate}
\usepackage[lmargin=71pt, tmargin=0.8in]{geometry}  %For centering solution box

% \chead{\hline} % Un-comment to draw line below header
\thispagestyle{empty} 
\usepackage{color}
\definecolor{light-gray}{gray}{0.95}

\usepackage{listings}
\lstset{
    basicstyle=\footnotesize\ttfamily,
    escapechar=¢,
    language=python,
    frame=single,
    frameround=tttt,
    showstringspaces=false,
    backgroundcolor=\color{light-gray}
}  %For removing header/footer from page 1


\newcommand*{\ipythonprompt}[1]{\makebox[0pt][r]{\textbf{In [#1]:}\hspace{1em}}}

\usepackage{xcolor}

\definecolor{codegreen}{rgb}{0,0.6,0}
\definecolor{codegray}{rgb}{0.5,0.5,0.5}
\definecolor{codepurple}{rgb}{0.58,0,0.82}
\definecolor{backcolour}{rgb}{0.95,0.95,0.92}

\lstdefinestyle{mystyle}{
    backgroundcolor=\color{light-gray},   
    commentstyle=\color{codegreen},
    keywordstyle=\color{magenta},
    numberstyle=\tiny\color{codegray},
    stringstyle=\color{codepurple},
    basicstyle=\ttfamily\footnotesize,
    breakatwhitespace=false,         
    breaklines=true,                 
    captionpos=b,                    
    keepspaces=true,                 
    numbers=left,                    
    numbersep=5pt,                  
    showspaces=false,                
    showstringspaces=false,
    showtabs=false,                  
    tabsize=2
}

\lstset{style=mystyle}




\begin{document}



\newtheorem{theorem}{Theorem}[section]
\newtheorem{problem}{Problem}
\newtheorem{proposition}{Proposition}[section]
\newtheorem{lemma}{Lemma}[section]
\newtheorem{corollary}[theorem]{Corollary}
\newtheorem{example}{Example}[section]
\newtheorem{definition}[problem]{Definition}

\newcommand{\BEQA}{\begin{eqnarray}}
\newcommand{\EEQA}{\end{eqnarray}}
\newcommand{\define}{\stackrel{\triangle}{=}}
\bibliographystyle{IEEEtran}
\raggedbottom
\setlength{\parindent}{0pt}
\providecommand{\mbf}{\mathbf}
\providecommand{\norm}[1]{\lVert#1\rVert}
\providecommand{\pr}[1]{\ensuremath{\Pr\left(#1\right)}}
\providecommand{\qfunc}[1]{\ensuremath{Q\left(#1\right)}}
\providecommand{\sbrak}[1]{\ensuremath{{}\left[#1\right]}}
\providecommand{\lsbrak}[1]{\ensuremath{{}\left[#1\right.}}
\providecommand{\rsbrak}[1]{\ensuremath{{}\left.#1\right]}}
\providecommand{\brak}[1]{\ensuremath{\left(#1\right)}}
\providecommand{\lbrak}[1]{\ensuremath{\left(#1\right.}}
\providecommand{\rbrak}[1]{\ensuremath{\left.#1\right)}}
\providecommand{\cbrak}[1]{\ensuremath{\left\{#1\right\}}}
\providecommand{\lcbrak}[1]{\ensuremath{\left\{#1\right.}}
\providecommand{\rcbrak}[1]{\ensuremath{\left.#1\right\}}}
\let\vec\mathbf

\newlist{mydesc}{description}{1} % create a new list called mydesc, of type "description"
\setlist[mydesc]{
  align=left, % use the align-format defined above
  leftmargin=0pt, % indentation for all the lines
  labelindent=1em, % horizontal space before label
  labelsep=0pt
   % horizontal space after label -- set to zero because we add space via "leftwithbar"
}



\begingroup  
    \centering
    
    \LARGE Weekly Report 3 - PCA\\[0.5em]
    
    \large Ganji Varshitha\par
    \large AI20BTECH11009\par
\endgroup
\rule{\textwidth}{0.4pt}
\pointsdroppedatright   %Self-explanatory
\printanswers
\newcommand\Solution{
  \textbf{Solution:}\\}
\newcommand{\myvec}[1]{\ensuremath{\begin{bmatrix}#1\end{bmatrix}}}
 %Replace "Ans:" with starting keyword in solution box

 \subsection*{Introduction}
PCA is a data-visualization and data-preprocessing technique for training data before learning a machine learning model.
It comes under dimensionality reduction.

\subsection*{Algorithm}
It can be seen as projecting the original data point vector into a lower dimensional space where information loss is minimum.
Let us assume projection of $\vec{x}$ in the direction of $\vec{w}$ be $\vec{z} = \vec{w}^\top \vec{x}$.
\\Maximising variance of $\vec{z}$ gives $\vec{w}$ such that all principal components are orthogonal to each other and $\norm{w}=1$  
\\
Let X be matrix where rows corresponding to data points and columns corresponding to features
\begin{algorithm}
\caption{PCA algorithm}\label{cap}
\begin{algorithmic}

\State Standardize the features with 0 mean and 1 variance. Call the new matrix as Z.



\State Calculate covariance of Z.
\State Calculate the eigenvalues and eigenvectors of covariance matrix of Z.
\State Sort the eigenvectors in decreasing order of their corresponding eigenvalues.
\State Select the optimal number of eigenvectors, say K.
\State Let P be a matrix with each column corresponds to sorted K eigenvectors. The reduced dataset matrix becomes transform of $(ZP)^\top$


\end{algorithmic}
\end{algorithm}

\subsection*{Code Snippets}
\begin{lstlisting}

def standardize(data):
  dataset = data.values
  A = dataset.T
  V = np.empty((A.shape))
  for i in range(A.shape[0]):
    for j in range(A.shape[1]):
      V[i,j] = (A[i,j] - np.mean(A[i]))/np.std(A[i])
  return V.T
\end{lstlisting} 
\begin{lstlisting}
# Standardizing the data
X = standardize(data)
# Calculating the covariance matrix
cov = np.cov(X.T)
# Finding eigenvalues and eigenvectors of cov matrix
eigen_values, eigen_vectors = np.linalg.eigh(cov)
print("Eigenvalues:\n",eigen_values)
print("Eigenvectors:\n",eigen_vectors)
\end{lstlisting}
\begin{lstlisting}
# Printing principal components
print("First principal component: \n",eigen_vectors[:,2])
print("Second principal component: \n",eigen_vectors[:,1])
print("Third principal component: \n",eigen_vectors[:,0])
\end{lstlisting}
\begin{lstlisting}
# Using two first principal components 
projmatrix = np.zeros((3,2))
projmatrix[:,0] = eigen_vectors[:,0]
projmatrix[:,1] = eigen_vectors[:,2]

# Reducing dataset to 2 dimensions:
X_red = (X@projmatrix).T
print(X_red)
\end{lstlisting}

\subsection*{Key points}
\begin{itemize}
\item It assumes correlation among the features. If the features are not correlated the algorithm may not be useful
\item Standardizing the features is necessary as the algorithm is sensitive to the scale.
\item It assumes linear relationship between features and does not work well for non linear relationships.
\item It is a statistical process which acts as a tool in EDA(Exploratory data analysis).
\item PCA is not robust to outliers.
\item Since the reduced data matrix has features as linear combination of original features, it lacks interpret-ability.
\item It helps in increasing model performance for high dimensional data regression problems and prevents over fitting.
\end{itemize}





\end{document}